\documentclass{article}
\usepackage{multido}
\usepackage{pst-3dplot}
\usepackage{pstricks-add}
\usepackage{animate}

\makeatletter
%%%%%%%%%%%%%%%%%%%%%%%%%%%%%%%%%%%%%%%%%%%%%%%%%%%%%%%%%%%%%%%%%%%%%%%%%%%%%%%%
%
% odesolve
%
% LaTeX command for integrating systems of first order ODEs using the Runge-
% Kutta method; values of the integration parameter `t' as well as the solution
% (= state) vectors are written to a text file
%
% Usage:
%
% \odesolve{filename}{output vector}{ta}{tb}{nodes}{initial cond.}{function}
% \odesolve*{filename}{output vector}{ta}{tb}{nodes}{initial cond.}{function}
%
%%%%%%%%%%%%%%%%%%%%%%%%%%%%%%%%%%%%%%%%%%%%%%%%%%%%%%%%%%%%%%%%%%%%%%%%%%%%%%%%
% #1: output filename for solution data
% #2: output vector format, e. g. `(t) 0 1'; specifies which data to be written
%     to #1; (t) (parentheses required) writes integration parameter to the data
%     file, 0, 1, 2, etc. specify the elements of the state vector to be written
% #3: start value of integration parameter (ta)
% #4: end value of integration parameter (tb)
% #5: number of output points (nodes), including ta and tb
% #6: initial condition vector; if empty, use state vector from last \odesolve
%     invocation
% #7: right hand side of ODE system; the function provided should pop the
%     elements of the current state vector from the operand stack and push the
%     first derivatives (right hand side of ODE system) back to it, the
%     integration parameter can be accessed using `t'
%
% \odesolve* --> computed data are appended to existing data file (arg. #1)
%                rather than overwriting it
%%%%%%%%%%%%%%%%%%%%%%%%%%%%%%%%%%%%%%%%%%%%%%%%%%%%%%%%%%%%%%%%%%%%%%%%%%%%%%%%
\def\odesolve{\@ifstar{\@odesolve[append]}{\@odesolve}}
\newcommand{\@odesolve}[8][]{%
  \def\append{false}%
  \def\filemode{w}%
  \ifthenelse{\equal{#1}{append}}{%
    \def\append{true}%
    \def\filemode{a}}{}%
  \def\initcond{}%
  \ifthenelse{\equal{#7}{}}{}{%
    \def\initcond{/laststate [#7] def}%
  }%
  \pstVerb{%
    /statefile (#2) (\filemode) file def
    /outvect [#3] def
    /t #4 def
    /tEnd #5 def
    /dt tEnd t sub #6\space 1 sub div def % step size
    /dt2 dt 2 div def % half step size
    \initcond %set initial state vector
    /xlength laststate length def % number of equations
    /xlength1 xlength 1 add def % number of equations plus 1
    /ODESET { cvx exec #8 xlength array astore } def %system of ODEs
    /addvect { % [1 2 3] [4 5 6] addvect => [5 7 9]
      cvx exec xlength1 -1 roll {xlength1 -1 roll add} forall
      xlength array astore
    } def
    /mulvect { % [1 2 3] 4 mulvect => [4 8 12]
      /mul cvx 2 array astore cvx forall xlength array astore
    } def
    /divvect { % [4 8 12] 2 divvect => [2 4 6]
      /div cvx 2 array astore cvx forall xlength array astore
    } def
    /RK { % performs one Runge-Kutta integration step
          % [ state vector x(t) ] RK => [ state vector x(t + dt) ]
      dup ODESET /k0 exch def
      t dt2 add /t exch def
      dup k0 dt2 mulvect addvect ODESET /k1 exch def
      dup k1 dt2 mulvect addvect ODESET /k2 exch def
      t dt2 add /t exch def
      dup k2 dt mulvect addvect ODESET /k3 exch def
      k0 k1 2 mulvect addvect k2 2 mulvect addvect k3 addvect
      6 divvect dt mulvect addvect
    } def
    /output { %output routine
      outvect {
           dup (t) eq {
             pop t 20 string cvs statefile exch writestring
           }{
             laststate exch get 20 string cvs statefile exch writestring
           } ifelse
           statefile (\space) writestring
      } forall
      statefile (\string\n) writestring
    } def
    \append\space not {output} if
    #6\space 1 sub {laststate RK /laststate exch def output} repeat
    statefile closefile
  }%
}
%%%%%%%%%%%%%%%%%%%%%%%%%%%%%%%%%%%%%%%%%%%%%%%%%%%%%%%%%%%%%%%%%%%%%%%%%%%%%%%%
\makeatother

\begin{document}
%
% Lorenz' set of differential equations
\pstVerb{
  /lorenz {
    %get elements of current state vector
    /varz exch def /vary exch def /varx exch def
    %
    10 vary varx sub mul                  %dx/dt
    varx 28 varz sub mul                  %dy/dt
    varx vary mul 8 3 div varz mul sub    %dz/dt
  } def
}%
%
%write timeline file
\newwrite\OutFile%
\immediate\openout\OutFile=lorenz.tln%
\multido{\iLorenz=0+1}{101}{%
  \immediate\write\OutFile{::\iLorenz x0}%
}%
\immediate\write\OutFile{::c,101}%
\multido{\iLorenz=102+1}{89}{%
  \immediate\write\OutFile{::\iLorenz}%
}%
\immediate\closeout\OutFile%
%
\begin{center}
\psset{unit=0.155,linewidth=0.5pt}%
\begin{animateinline}[
  timeline=lorenz.tln,
  controls,poster=last,
  begin={\begin{pspicture}(-39,-13)(39,60)},
  end={\end{pspicture}}
]{10}%
  \psset{Alpha=120,Beta=20}%
  \pstThreeDCoor[xMax=33,yMax=33,zMax=55,linecolor=black]%
\newframe%
  \pstVerb{/laststate [3 15 1] def}%
  \multiframe{100}{rtZero=0+0.25,rtOne=0.25+0.25}{% t0, t1
    \odesolve{lorenz.dat}{0 1 2}{\rtZero}{\rtOne}{26}{}{lorenz}%
    \pstVerb{/infile (lorenz.dat) (r) file def}%
    \parametricplotThreeD[%
      plotstyle=line,xPlotpoints=26](0,0){infile 80 string readline pop cvx exec}%
  }%
\newframe%
  \odesolve{lorenz.dat}{0 1 2}{0}{25}{2501}{3 15 1}{lorenz}%
  \multiframe{90}{rAlpha=116+-4}{%
    \psset{Alpha=\rAlpha,Beta=20}%
    \pstThreeDCoor[xMax=33,yMax=33,zMax=55,linecolor=black]%
    \pstVerb{/infile (lorenz.dat) (r) file def}%
    \parametricplotThreeD[%
      plotstyle=line,xPlotpoints=2501](0,0){infile 80 string readline pop cvx exec}%
  }%
\end{animateinline}
\end{center}

\end{document}